
\section{The Convolution Algorithm}
\label{sec:convolution}

\subsection*{Theory and Exercises}

\Opensolutionfile{hint}
\Opensolutionfile{ans}

In view of the later algorithm, it is more consistent to write
$p_i(k|N)=\pi(n_i=k)$ for the marginal probability that station $i$
contains $k$ jobs, under the assumption that the network contains $N$
jobs. 

Assume we have computed  $p_i(k|N)$  with Eq.3.15, i.e., with the recursion
\begin{equation*}
  p_i(k|N) = f_i(k)\frac{G(M\setminus\{i\}, N-k)}{G(M,N)}.
\end{equation*}
We can then compute the rest of the performance measures as 
\begin{align*}
  \TH_i &= \mu_i\sum_{k=1}^N p_i(k|N), \\
  \E W &= \frac{N}{TH_0}, \\
\E L_i &= \sum_{k=1} k p_i(k|N), \\
\E W_i &= \E L_i / V_i, \text{ by Little's law}. \\
\end{align*}
The rest of the equations in Section 3.1.2 are interesting, but not
necessary for an implementation in computer code. 

\begin{exercise}
  Explain Zijm.Eq.3.13.
  \begin{solution}
    It follows directly from the fact that $G(0,N)$ is a normalization
    constant. When there is just one station, i.e., station 0, all jobs are at that station. Hence, $\pi_0(N)=1$. From Zijm.Eq.3.1--Zijm.Eq.3.3, it follows that $f_0(N)=(V_0/\mu_0)^N$, so that $G(0,N)$ must be $1/f_0(N)$. Finally, since $V_0=1$ always, we get $G(0,N)=\mu_0^{-N}$.
  \end{solution}
\end{exercise}

\begin{comment}
\begin{exercise}
TBD. Relate the following to the arrival theorem below.  The derivation of
Zijm.Eq.3.8 is particularly interesting. Rewrite Zijm.Eq.3.5 as
\begin{equation*}
  \pi_{M,N}(N_i=k) = f_i(k)\frac{G(M\setminus\{i\}, N-k)}{G(M,N)},
\end{equation*}
to denote, \emph{ for a network with $M$ stations and $N$ jobs}, the
(marginal) stationary distribution that station $i$ contains $k$ jobs. With this notation
\begin{equation*}
  \begin{split}
    \TH_i 
&= \frac{V_i}{G(M,N)} \sum_{k=1}^{N}f_i(k-1) G(M\setminus\{i\}, N-k) \\
&= \frac{V_i}{G(M,N)} \sum_{k=0}^{N-1}f_i(k-1) G(M\setminus\{i\}, N-k-1) \\
&= V_i \sum_{k=0}^{N-1}f_i(k-1) \frac{G(M\setminus\{i\}, N-k-1)}{G(M,N)} \\
&= \sum_{k=0}^{N-1} (V_i \pi_{M,N-1}(N_i=k-1)) \\
&= V_i \frac{G(M,N-1)}{G(M,N)}.
  \end{split}
\end{equation*}
The second to last equation is the most interesting in fact. This says
that the throughput of station $i$ is (the sum of) the visit ratio
$V_i$ times the stationary probabilities of a system with one job
less, i.e., $N-1$, and that this job is taken away from station $i$,
i.e., $N_i=k-1$.  
  \begin{solution}
    TBD.
  \end{solution}
\end{exercise}


\begin{exercise}[use=false]
  Show in the one of the examples below how to compute $\E L_i$
  \begin{solution}
    TBD.
  \end{solution}
\end{exercise}
\end{comment}


\begin{exercise}
Zijm.Ex.3.1.6
\begin{solution}
  This is related to the famous ball/boxes problem in Feller. Given
  $n$ (identical) balls and $m$ boxes, in how many configurations can
  the balls be distributed over the boxes? One nice way to get the
  answer is to use recursion. (I leave it to you to figure this
  out). The other is this: imagine that all boxes are put next to each
  other. Then we can imagine that two neighboring boxes are separated
  by a stick. Since there are $m$ boxes, $m-1$ sticks suffice to
  separate all boxes. This observation allows us to reduce the
  balls/boxes problem to figuring out in how many ways all sticks and
  balls can be mixed. Since the balls and sticks are all identical,
  and since we have $n+m-1$ objects ($n$ balls and $m-1$ sticks), these objects can be distributed in 
  \begin{equation*}
{n+m-1 \choose n} =     {n+m-1 \choose m-1}
  \end{equation*}
  ways.  (BTW, this question relates directly to the number of edge
  points that occur as potential solutions in a linear programming
  problem.)

  Now interpret the boxes as stations and the jobs as balls. Since we
  have $M+1$ stations, we get the result.
\end{solution}
\end{exercise}

\begin{exercise}
Zijm.Ex.3.1.7
\begin{solution}
  If the production rate at station $i$ is infinitely large, the time
  a job stays at station $i$ must be 0. Thus, the only `activity' of
  station $i$ would be to distribute all arriving jobs between its
  receiving stations, without any delay.

  Thus, in both cases (i.e., setting the rate to infinity or defining
  $P_{j,k}' = P_{ji}P_{ik} + P_{j,k} $) it comes down to
  `short-circuiting' node $i$.
\end{solution}
\end{exercise}


\begin{exercise}
Zijm.Ex.3.1.8
\begin{solution}
From Zijm.Eq.3.3 for $c_i=1$ we see that $f_i(n_i) = (V_i/\mu_i)^{n_i}$, since $\min\{c_i, k\} = \min\{1, k\} = 1$. Now using Zijm.Eqs 3.5 and 3.6,
  \begin{equation*}
    \begin{split}
      \P{n_i \geq k} 
&= \sum_{m=k}^N \pi(n_i = m) \\
&= \frac1{G(M,N)}\sum_{m=k}^N f_i(m)G(M\setminus\{i\}, N-m) \\
&= \frac1{G(M,N)}\sum_{m=k}^N \left(\frac{V_i}{\mu_i}\right)^{m}G(M\setminus\{i\}, N-m) \\
&= \frac1{G(M,N)}\left(\frac{V_i}{\mu_i}\right)^{k}\sum_{m=0}^{N-k} \left(\frac{V_i}{\mu_i}\right)^{m}G(M\setminus\{i\}, N-k-m) \\
&= \frac{f_i(k)}{G(M,N)}\sum_{m=0}^{N-k} \left(\frac{V_i}{\mu_i}\right)^{m}G(M\setminus\{i\}, N-k-m) \\
&= \frac{f_i(k)}{G(M,N)}G(M, N-k).
    \end{split}
  \end{equation*}
  For $c>1$, the expression for $f_i$ is less simple, and the above
  derivation is not possible anymore.
\end{solution}
\end{exercise}

\begin{exercise}
  Zijm.Ex.3.1.9. 
  \begin{hint}
This problem forces you to really trace down all
  relevant definitions. Please try to do it yourself first. It is
  harder than you might think initially.
  \end{hint}
\begin{solution}
  The last time I solved this problem was in 2001. My answer below
  reflects my train of thought, including the mistakes. 

  My first guess is that
  $\pi_i(1) = \pi((0,0,\ldots, 1,\ldots, 0))= 1/M$, where
  $(0,0,\ldots, 1,\ldots, 0)$ is the vector with a $1$ at the $i$-th
  position and $0$ elsewhere. It must be that $\pi(n)=0$ for $n>1$,
  because there is just one job in the system.

Let us check Zijm.Eq.3.5:
\begin{equation*}
  \pi(n_i=k) = f_i(k) \frac{G(M\setminus\{i\}, N-k)}{G(M,n)},
\end{equation*}
and  Zijm.3.3:
\begin{equation*}
  f_i(n_i) = \frac{1}{\Pi_{k=1}^{n_i} \min\{k, c_i\}} \left(\frac{V_i}{\mu_i}\right)^{n_i}.
\end{equation*}
We take $N=1$ in Zijm.3.3. Since $\sum_{i} n_i = N$, and $N=1$,
$n_i\in\{0,1\}$. Thus,
\begin{equation*}
  f_i(0) = 1.
\end{equation*}
This is a bit tricky, since the product $\Pi$ over $k$ is empty. Such empty products are defined to be 1. Next, whatever $c_i$, $\min\{1,c_i\}=1$. Thus, 
\begin{equation*}
  f_i(1) = \frac 11 \left(\frac{V_i}{\mu_i}\right)^{1}.
\end{equation*}
Now we need the visit ratios $V_i$. The visit ratios must be one since
all jobs move from station $n$ to $n+1$, in a circle. BTW, this is the
same as Zijm.Question 3.1.2.  Thus, 
\begin{equation*}
  f_i(1) = \frac{1}{\mu_i}.
\end{equation*}

Now I suddenly see that my initial guess about $\pi_i(1) = \pi(n_i=1)$
must be wrong. Rather, it has to be proportional to the service times!
So, I'll revise my guess to
$\pi_i(1) = \mu_i^{-1}/\sum_{j=0}^M \mu_i^{-1}$. 

I also see now  I can better use Zijm.3.6:
\begin{equation*}
  G(M,N)=\sum_{k=0}^N f_i(k)G(M\setminus\{i\} , N-k).
\end{equation*}
So,
$G(M,1) = f_1(0)G(M\setminus\{1\}, 1) + f_1(1)G(M\setminus\{1\},
0)$.
Now, when there is no job in the network, there can be only
configuration, hence $G(M,0)=1$ for all $M$. Thus,
$G(M,1) = G(M\setminus\{1\}, 1) + 1/\mu_1$, where I use the above
results for $f_i(0)$ and $f_i(1)$. But this holds for any station $i$,
so
\begin{equation*}
  G(M,1) = G(M\setminus\{1\}, 1) + 1/\mu_1 = G(M\setminus\{i\}, 1) + 1/\mu_i,
\end{equation*}
and so the only choice left for $G(M,N) = \sum_i \mu_i^{-1}$, just as
I guessed after my repair.

I still need to fill in Zijm.3.5. From the above I learn that
$f_i(0) G(M\setminus\{i\}, 1-0) = G(M\setminus\{i\}, 1)$ and
$f_i(1) G(M\setminus\{i\}, 0) = \mu_i^{-1}$. 
Thus, 
\begin{align*}
  \pi_i(0) &=  f_i(0) \frac{G(M\setminus\{i\}, 1-0)}{G(M,1)} = \frac{G(M\setminus\{i\}, 1)}{G(M,N)} = 
\frac{\sum_{j\neq i} \mu_m^{-1}}{\sum_{j=1}^M \mu_j^{-1}}.
\\
  \pi_i(1) &=  f_i(1) \frac{G(M\setminus\{i\}, 0)}{G(M,1)} = \frac{\mu_i^{-1}}{G(M,N)} = \frac{\mu_i^{-1}}{\sum_{j=1}^M \mu_j^{-1}}.
\end{align*}

I am happy to see that my second, less naive, guess comes out
right. So why then must this be the intuitive answer? Lets consider a
trivial network, with one load/unload node, and two single-server
stations, one with a very fast server, and the other with a very slow
server. Then the job must be at the slow server most of the time. The
most natural guess is then that the time spent at a station, provided
the visit ratios are one and there is one job, is proportional to its
service time.
\end{solution}
\end{exercise}


\begin{exercise}
Zijm.Ex.3.1.10. Assume that the visit ratios are 1.
\begin{hint}
 Compare Zijm.Question 3.1.5.
\end{hint}
\begin{solution}
  $G(2,2)$ means that we have $2$ stations and $2$ jobs. Lets start
  from Zijm.Eq.3.6 but now applied to node $M=2$:
\begin{equation*}
  G(2,2)=\sum_{k=0}^2 f_2(k)G(2\setminus\{2\} , 2-k),
\end{equation*}
and use Zijm.Eq.3.13
\begin{equation*}
  G(0,2) = \mu_0^{-2}.
\end{equation*}

\begin{equation*}
  \begin{split}
  G(2,2)
&=\sum_{k=0}^2 f_2(k)G(2\setminus\{2\} , 2-k) = \sum_{k=0}^2 f_2(k)G(1 , 2-k) \\
& = f_2(0) G(1,2) + f_2(1)G(1,1) + f_2(2) G(1,0) \\
& =  G(1,2) + \mu_2^{-1}G(1,1) + \mu_2^{-2} G(1,0) \\
  \end{split}
\end{equation*}
where we use Zijm.Eq.3.3 to see that $f_2(k) = \mu_2^{-k}$ (realize
again that we took the visit ratios to be one).  Thus, we need
$G(1,\cdot)$.
\begin{align*}
  G(1,2)
&=\sum_{k=0}^2 f_1(k)G(1\setminus\{1\} , 2-k) = \sum_{k=0}^2 f_1(k)G(0 , 2-k) \\
& = f_1(0) G(0,2) + f_1(1) G(0,1)  + f_1(2)G(0,0), \\
& = f_1(0) \mu_0^{-2} + f_1(1) \mu_0^{-1} + f_1(2), \\
&= \mu_0^{-2} + \mu_1^{-1} \mu_0^{-1} + \mu_1^{-2},
\end{align*}
where we use that $f_1(k) = \mu_1^{-k}$ and $G(M,0)=1$ and $G(0,N) = \mu_0^{-N}$. 
Next, 
\begin{align*}
G(1,1) &= f_1(0)G(0,1) + f_1(1)G(0,0) = \mu_0^{-1} + \mu_1^{-1},\\
G(1,0) &= 1.
\end{align*}
Filling in the above, we get
\begin{equation*}
  G(2,2) = \mu_0^{-2}+\mu_1^{-1}+\mu_2^{-2} + (\mu_0\mu_1)^{-1} + (\mu_1\mu_2)^{-1} + (\mu_0\mu_2)^{-1}.
\end{equation*}
\end{solution}
\end{exercise}


\begin{exercise}
Zijm.Ex.3.1.11
\begin{solution}
  With the code I can check the results of the previous problem.  I'll
  discuss the code in steps. I encourage you to study this example very carefully. Your
  programming skills will improve quite a bit. 

We first need a few libraries. 

\begin{pyconsole}
from functools import lru_cache
import numpy as np
from numpy.linalg import eig
  
\end{pyconsole}

Since the convolution algorithm is recursive I use caching to store
the results of intermediate computations. The storage is taken care of
by the, so-called, decorator \texttt{lru\_cache}. This programming
concept is typically called \emph{memoization}; there is a tutorial in
R on memoization with Fibonacci sequences (see google) if you prefer
to study this idea in R. Memoization is an important concept to learn
as it can give an enormous computational speed up, often from
exponential to linear complexity.

The data is this:

\begin{pyconsole}
mu = np.array([2, 1, 3])
c = np.array([1, 1, 1]) # single server stations
P = np.matrix([
    [0, 1, 0],
    [0, 0, 1],
    [1, 0, 0]
]
)
  
\end{pyconsole}

I now compute the visit ratios $V$. This vector is the left
eigenvector of the routing matrix~$P$ with eigenvalue $1$, i.e.,
$V=VP$. Since $P$ is a routing matrix it can be proven that only one
eigenvalue is equal to 1, and that for all the other eigenvalues the
real part is less than 1, that is, if $v$ is such an eigenvalue, then
$\Re(v) < 1$. (If you are interested to understand why, check
the Perron-Frobenius theorem.) Thus, the visit ratio is the only
eigenvector related to the eigenvalue $1$, and, moreover, the
eigenvalue 1 is the largest of all eigenvalues.

Compute the eigenvalues and eigenvectors:

\begin{pyconsole}
v, w = eig(P.T)
\end{pyconsole}

Get the index of the  eigenvalue with the largest real part, i.e., the  eigenvalue 1:
\begin{pyconsole}
x = v.real.argmax() 
\end{pyconsole}

V is the eigenvector with eigenvalue 1; take the real part of the entries of $V$:
\begin{pyconsole}
V = w[:,x].real 
\end{pyconsole}

Normalize so that $V_0 = 1$. 

\begin{pyconsole}
V = V / V[0] 
V
  
\end{pyconsole}
This is what we expected.

It follows from Zijm.Eq.3.3 that 
\begin{equation*}
  f_i(n_i) = 
  \begin{cases}
\frac{V_i}{\mu_i} \frac1{\min\{n_i, c_i\}} f_i(n_i-1), & \text{ if } n_i > 0, \\
1, & \text{ if } n_i =0.
  \end{cases}
\end{equation*}

\begin{pyconsole}
@lru_cache(maxsize=None)
def f(i, n_i):
    if n_i == 0:
        return 1
    return V[i] / mu[i] / min(n_i, c[i]) * f(i, n_i - 1)
  
\end{pyconsole}

Note that the cache stores the results of intermediate computations.

Finally the convolution algorithm.

\begin{pyconsole}
@lru_cache(maxsize=None) 
def G(m, n):
    if n == 0:
        return 1
    if m == 0:
        return 1 / mu[0]**n
    return sum(f(m, k) * G(m - 1, n - k) for k in range(n + 1))
  
\end{pyconsole}

Observe how simple this implementation is compared to the
specification in the text. I also find it much easier to read. 

The result of the previous example must be 

\begin{pyconsole}
G(2, 2)
\end{pyconsole}

Testing code is crucial, just as crucial as testing formulas. The
analytic result of the previous question becomes in code:

\begin{pyconsole}
def test():
    res = 1 / mu[0]**2 + 1 / mu[1]**2 + 1 / mu[2]**2
    res += 1 / mu[0] * 1 / mu[1] + 1 / mu[0] * \
        1 / mu[2] + 1 / mu[1] * 1 / mu[2]
    print(res)

test()
  
\end{pyconsole}

The  outcomes of both procedures are the same. 

When $M$ and $N$ are in the order of 1000, then I guess that the
algorithm will not be that useful anymore. Even model the production
situation as a closed queueing network will, typically, use its
value. For instance, the convolution algorithm requires the processing
rates of all the stations. Who on earth will assemble the processing
rates of 1000 machines? Often, only the machines with the highest load
are important the analyze, as these machines are the bottleneck. The
rest can be, more or less, neglected, or modeled with very simple
models.

\end{solution}
\end{exercise}

\begin{exercise}
Zijm.Ex.3.1.12
\begin{solution}
  Because then the form of $f_i(n_i)\neq (V_i/\mu_i)^{n_i}$, but much
  less simple.

We can try the recursion of Example 3.3.

\begin{pyconsole}
mu = np.array([2, 1, 3])
V = np.array([1, 1, 1])


@lru_cache(maxsize=None)
def G(M, N):
    if N == 0:
        return 1
    if M == 0:
        return 1 / mu[M]**N
    return G(M - 1, N) + G(M, N - 1) * V[M] / mu[M]

print(G(2, 2))
  
\end{pyconsole}
\end{solution}
\end{exercise}





\Closesolutionfile{hint}
\Closesolutionfile{ans}

\opt{solutionfiles}{
\subsection*{Hints}
\input{hint}
\subsection*{Solutions}
\input{ans}
}


%\clearpage
